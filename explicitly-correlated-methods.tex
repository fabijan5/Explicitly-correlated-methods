\documentclass[journal=jacsat]{achemso}

%%%%%%%%%%%%%%%%%%%%%%%
% Includes various packages for extra functionality
%%%%%%%%%%%%%%%%%%%%%%%
%\usepackage{modiagram}	% Molecular orbital diagrams

\usepackage{mathtools}		% Make pretty formulas
\usepackage{fixltx2e}		% Allow subscripts
\usepackage{graphicx}		% Improved graphics
\usepackage{bm} 			% Bold math symbols 
\usepackage{txfonts} 		% Allow times-like fonts in equations
\usepackage{mathpazo} 		% Palatino fonts for equations
\usepackage{dcolumn}		% Align tabular columns on decimal points
\usepackage{footmisc}		% Foot note options
\usepackage{mathrsfs}		% RSFS fonts in equations
\usepackage{mhchem}
\usepackage{braket}
\usepackage{commath}
\usepackage[T1]{fontenc}
\usepackage{currvita}
\numberwithin{equation}{section}
\usepackage{simplewick}

%\usepackage[polish]{babel}	% Select language standard
%\usepackage{natbib}		% Support for various natbib BibTex styles


% This is for use of Times New Roman
\usepackage[T1]{fontenc}
\usepackage{times}

\newcommand{\T}[4]{T^{#1 #2}_{#3 #4}}
\renewcommand{\O}[1]{$\mathcal{O}\left( #1 \right)$}

\renewcommand{\thefootnote}{\fnsymbol{footnote}}
\renewcommand{\baselinestretch}{2} % This is for the use of double space
%\setkeys{acs}{articletitle = true} % Include title in bibliography

%%%%%%%%%%%%%%%%%%%%%%%
% Start document
%%%%%%%%%%%%%%%%%%%%%%%

%%%%%%%%%%%%%%%%%%%%%%%
% Title page
%%%%%%%%%%%%%%%%%%%%%%%

\author{Fabijan Pavo\v{s}evi\'{c}}
\email{fabijan@vt.edu}
\affiliation[Virginia Tech]
       {Department of Chemistry, Virginia Tech, Blacksburg, VA}
\title{Explicitly correlated methods}
\date{September 6, 2013}

\SectionNumbersOn
\begin{document}

\maketitle
%%%%%%%%%%%%%%%%%%%%%%%
% Paper
%%%%%%%%%%%%%%%%%%%%%%%
\newpage
\tableofcontents
\newpage
\section{Rayleigh-Schr\"{o}dinger Perturbation theory}
The exact Schr\"{o}dinger equation of the form
\begin{align}
\label{eq:schrodinger_eq}
\hat{H}\ket{\Psi_{i}}=E_{i}\ket{\Psi_{i}}
\end{align}
is too complex to be solved exactly. One way to find an approximate solution to Schr\"{o}dinger equation is via perturbation theory in which we divide Hamiltonian operator into zeroth- and first-order contributions as:
\begin{align}
\hat{H}=\hat{H}^{(0)} + \lambda\hat{H}^{(1)}
\end{align}
where $ \hat{H}^{(1)}$ represents a small perturbation to the system while for the  $\hat{H}^{(0)}$ we have and exact solution. Furthermore, we can expand energy and wavefunction as:
\begin{align}
E_{i} &= E_{i}^{(0)} + \lambda E_{i}^{(1)} + \lambda^{2} E_{i}^{(2)} + ...\\
\ket{\Psi_{i}} &= \ket{\Psi_{i}^{(0)}} + \lambda \ket{\Psi_{i}^{(1)}} + \lambda^{2} \ket{\Psi_{i}^{(2)}} + ...
\end{align}
Insertion of such defined Hamiltonian, wavefunction and energy into Eq. \ref{eq:schrodinger_eq}, we obtain
\begin{align}
(\hat{H}^{(0)} + \lambda\hat{H}^{(1)})(\ket{\Psi_{i}^{(0)}} + \lambda \ket{\Psi_{i}^{(1)}} + \lambda^{2} \ket{\Psi_{i}^{(2)}} + ...)=\\
(E_{i}^{(0)} + \lambda E_{i}^{(1)} + \lambda^{2} E_{i}^{(2)} + ...)
(\ket{\Psi_{i}^{(0)}} + \lambda \ket{\Psi_{i}^{(1)}} + \lambda^{2} \ket{\Psi_{i}^{(2)}} + ...)
\end{align}
Collecting the terms with the same order of $\lambda$ gives:
 \begin{align}
\lambda^{0}:& \hat{H}^{(0)}\ket{\Psi_{i}^{(0)}}=E_{i}^{(0)}\ket{\Psi_{i}^{(0)}}\\
\lambda^{1}:&  \hat{H}^{(0)}\ket{\Psi_{i}^{(1)}} + \hat{H}^{(1)}\ket{\Psi_{i}^{(0)}}=E_{i}^{(0)}\ket{\Psi_{i}^{(1)}}+E_{i}^{(1)}\ket{\Psi_{i}^{(0)}}\\
\lambda^{n}:&  \hat{H}^{(0)}\ket{\Psi_{i}^{(n)}} + \hat{H}^{(1)}\ket{\Psi_{i}^{(n-1)}}=\sum_{k=0}^{n}E_{i}^{(k)}\ket{\Psi_{i}^{(n-k)}}
\end{align}

In order to obtain $n-th$ order energy expression, we can project $\lambda^{n}$ equation on the right hand side by $\bra{\Psi_{i}^{(0)}}$. After doing some straight forward algebra, we arrive at the first- and second order energy correction:
 \begin{align}
E_{i}^{(1)}&=\bra{\Psi_{i}^{(0)}}\hat{H}^{(1)}\ket{\Psi_{i}^{(0)}}\\
E_{i}^{(2)}&=\bra{\Psi_{i}^{(0)}}\hat{H}^{(1)}\ket{\Psi_{i}^{(1)}}
\end{align}

Finally, in this section we define second-order Hylleraas functional as:
 \begin{align}
H^{(2)}=2Re\bra{\Psi_{i}^{(0)}}\hat{H}^{(1)}-E^{(1)}\ket{\Psi_{i}^{(1)}}+\bra{\Psi_{i}^{(1)}}\hat{H}^{(0)}-E^{(0)}\ket{\Psi_{i}^{(1)}}
\end{align} 

\section{Second Quantization}
We define a creation operator by its action on a Slater determinant:
 \begin{align}
a_{p}^{\dagger}\ket{\phi_{q}...\phi_{r}}=\ket{\phi_{p}\phi_{q}...\phi_{r}}
\end{align} 
In the same way, we define annihilation operator: 
\begin{align}
a_{p}\ket{\phi_{p}\phi_{q}...\phi_{r}}=\ket{\phi_{q}...\phi_{r}}
\end{align} 
Slater determinant can be written as a chain of creation operators acting on the true (genuine) vacuum state ($\ket{vac}$):
\begin{align}
a_{p}^{\dagger}a_{q}^{\dagger}...a_{r}^{\dagger}\ket{vac}=\ket{\phi_{p}\phi_{q}...\phi_{r}}.
\end{align} 
Action of the annihilation operator onto vacuum state is:
\begin{align}
a_{p}\ket{vac}=0.
\end{align} 
The creation and annihilation operators obey anti commutation rules which can be derived from:
\begin{align}
a_{q}^{\dagger}a_{p}^{\dagger}\ket{vac}=\ket{\phi_{q}\phi_{p}}=-\ket{\phi_{p}\phi_{q}}=-a_{p}^{\dagger}a_{q}^{\dagger}\ket{vac}
\end{align} 
giving the firs anti commutation relation:
\begin{align}
[a_{q}^{\dagger},a_{p}^{\dagger}]_{+}=0
\end{align} 
Analogous relation can be obtained by taking complex conjugated for of the previous anti commutation relation giving:
\begin{align}
[a_{q},a_{p}]_{+}=0
\end{align} 
The final anti commutation relation may be obtained by considering:
\begin{align}
(a_{p}a_{p}^{\dagger}+a_{p}^{\dagger}a_{p})\ket{\phi_{q}...\phi_{r}}=a_{p}a_{p}^{\dagger}\ket{\phi_{q}...\phi_{r}}=\ket{\phi_{q}...\phi_{r}}
\end{align} 
giving
\begin{align}
a_{p}a_{p}^{\dagger}+a_{p}^{\dagger}a_{p}=1
\end{align} 
and 
\begin{align}
&(a_{p}a_{q}^{\dagger}+a_{q}^{\dagger}a_{p})\ket{\phi_{p}...\phi_{r}}=a_{p}a_{q}^{\dagger}\ket{\phi_{p}...\phi_{r}}+a_{q}^{\dagger}a_{p}\ket{\phi_{p}...\phi_{r}}=\\
&=a_{p}\ket{\phi_{q}\phi_{p}...\phi_{r}}+a_{q}^{\dagger}\ket{...\phi_{r}}=-\ket{\phi_{q}...\phi_{r}}+\ket{\phi_{q}...\phi_{r}}=0
\end{align} 
giving the 
\begin{align}
a_{p}a_{q}^{\dagger}+a_{q}^{\dagger}a_{p}=0
\end{align} 
The two expressions can be written using the one expression as:
\begin{align}
[a_{p},a_{q}^{\dagger}]_{+}=\delta_{pq}
\end{align} 

\subsection{Wick's theorem} 
Contraction for the product of two cration/annihalation operators is defined as:
\begin{align}
\contraction{}{A}{B}{}
AB\equiv AB - \{AB\}
\end{align}
where $\{AB\}$ denotes a normal ordered string. Normal ordered string with the respect to true vacuum is in which all annihilation operators are on the right of all creation operators.
An example of normal ordered string is:
\begin{align}
a_{q}^{\dagger}a_{p}\ket{vac}=0\\
a_{p}\ket{vac}=0
\end{align}  
In order to simplify the notation, we define creation and annihilation operators as:
\begin{align}
a_{p}^{\dagger}=a^{p} \\
a_{p}=a_{p}\\
a_{p}^{\dagger}a_{q}=a_{q}^{p}\\
a_{p}^{\dagger}a_{q}^{\dagger}a_{r}a_{s}=a^{pq}_{sr}\\
\end{align} 


\end{document}
