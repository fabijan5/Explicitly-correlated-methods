\documentclass[journal=jacsat]{achemso}

%%%%%%%%%%%%%%%%%%%%%%%
% Includes various packages for extra functionality
%%%%%%%%%%%%%%%%%%%%%%%
%\usepackage{modiagram}	% Molecular orbital diagrams

\usepackage{mathtools}		% Make pretty formulas
\usepackage{fixltx2e}		% Allow subscripts
\usepackage{graphicx}		% Improved graphics
\usepackage{bm} 			% Bold math symbols 
\usepackage{txfonts} 		% Allow times-like fonts in equations
\usepackage{mathpazo} 		% Palatino fonts for equations
\usepackage{dcolumn}		% Align tabular columns on decimal points
\usepackage{footmisc}		% Foot note options
\usepackage{mathrsfs}		% RSFS fonts in equations
\usepackage{mhchem}
\usepackage{braket}
\usepackage{commath}
\usepackage[T1]{fontenc}
\usepackage{currvita}
\numberwithin{equation}{section}
\usepackage{simplewick}

%\usepackage[polish]{babel}	% Select language standard
%\usepackage{natbib}		% Support for various natbib BibTex styles


% This is for use of Times New Roman
\usepackage[T1]{fontenc}
\usepackage{times}

\newcommand{\T}[4]{T^{#1 #2}_{#3 #4}}
\renewcommand{\O}[1]{$\mathcal{O}\left( #1 \right)$}

\renewcommand{\thefootnote}{\fnsymbol{footnote}}
\renewcommand{\baselinestretch}{2} % This is for the use of double space
%\setkeys{acs}{articletitle = true} % Include title in bibliography

%%%%%%%%%%%%%%%%%%%%%%%
% Start document
%%%%%%%%%%%%%%%%%%%%%%%

%%%%%%%%%%%%%%%%%%%%%%%
% Title page
%%%%%%%%%%%%%%%%%%%%%%%

\author{Fabijan Pavo\v{s}evi\'{c}}
\email{fabijan@vt.edu}
\affiliation[Virginia Tech]
       {Department of Chemistry, Virginia Tech, Blacksburg, VA}
\title{Explicitly correlated methods}
\date{September 6, 2013}

\SectionNumbersOn
\begin{document}

\maketitle
%%%%%%%%%%%%%%%%%%%%%%%
% Paper
%%%%%%%%%%%%%%%%%%%%%%%
\newpage
\tableofcontents
\newpage
\section{Rayleigh-Schr\"{o}dinger Perturbation theory}
The exact Schr\"{o}dinger equation of the form
\begin{align}
\label{eq:schrodinger_eq}
\hat{H}\ket{\Psi_{i}}=E_{i}\ket{\Psi_{i}}
\end{align}
is too complex to be solved exactly. One way to find an approximate solution to Schr\"{o}dinger equation is via perturbation theory in which we divide Hamiltonian operator into zeroth- and first-order contributions as:
\begin{align}
\hat{H}=\hat{H}^{(0)} + \lambda\hat{H}^{(1)}
\end{align}
where $ \hat{H}^{(1)}$ represents a small perturbation to the system while for the  $\hat{H}^{(0)}$ we have and exact solution. Furthermore, we can expand energy and wavefunction as:
\begin{align}
E_{i} &= E_{i}^{(0)} + \lambda E_{i}^{(1)} + \lambda^{2} E_{i}^{(2)} + ...\\
\ket{\Psi_{i}} &= \ket{\Psi_{i}^{(0)}} + \lambda \ket{\Psi_{i}^{(1)}} + \lambda^{2} \ket{\Psi_{i}^{(2)}} + ...
\end{align}
Insertion of such defined Hamiltonian, wavefunction and energy into Eq. \ref{eq:schrodinger_eq}, we obtain
\begin{align}
(\hat{H}^{(0)} + \lambda\hat{H}^{(1)})(\ket{\Psi_{i}^{(0)}} + \lambda \ket{\Psi_{i}^{(1)}} + \lambda^{2} \ket{\Psi_{i}^{(2)}} + ...)=\\
(E_{i}^{(0)} + \lambda E_{i}^{(1)} + \lambda^{2} E_{i}^{(2)} + ...)
(\ket{\Psi_{i}^{(0)}} + \lambda \ket{\Psi_{i}^{(1)}} + \lambda^{2} \ket{\Psi_{i}^{(2)}} + ...)
\end{align}
Collecting the terms with the same order of $\lambda$ gives:
 \begin{align}
\lambda^{0}:& \hat{H}^{(0)}\ket{\Psi_{i}^{(0)}}=E_{i}^{(0)}\ket{\Psi_{i}^{(0)}}\\
\lambda^{1}:&  \hat{H}^{(0)}\ket{\Psi_{i}^{(1)}} + \hat{H}^{(1)}\ket{\Psi_{i}^{(0)}}=E_{i}^{(0)}\ket{\Psi_{i}^{(1)}}+E_{i}^{(1)}\ket{\Psi_{i}^{(0)}}\\
\lambda^{n}:&  \hat{H}^{(0)}\ket{\Psi_{i}^{(n)}} + \hat{H}^{(1)}\ket{\Psi_{i}^{(n-1)}}=\sum_{k=0}^{n}E_{i}^{(k)}\ket{\Psi_{i}^{(n-k)}}
\end{align}

In order to obtain $n-th$ order energy expression, we can project $\lambda^{n}$ equation on the right hand side by $\bra{\Psi_{i}^{(0)}}$. After doing some straight forward algebra, we arrive at the first- and second order energy correction:
 \begin{align}
E_{i}^{(1)}&=\bra{\Psi_{i}^{(0)}}\hat{H}^{(1)}\ket{\Psi_{i}^{(0)}}\\
E_{i}^{(2)}&=\bra{\Psi_{i}^{(0)}}\hat{H}^{(1)}\ket{\Psi_{i}^{(1)}}
\end{align}

Finally, in this section we define second-order Hylleraas functional as:
 \begin{align}
H^{(2)}=2Re\bra{\Psi_{i}^{(0)}}\hat{H}^{(1)}-E^{(1)}\ket{\Psi_{i}^{(1)}}+\bra{\Psi_{i}^{(1)}}\hat{H}^{(0)}-E^{(0)}\ket{\Psi_{i}^{(1)}}
\end{align} 

\section{Second Quantization}
We define a creation operator by its action on a Slater determinant:
 \begin{align}
a_{p}^{\dagger}\ket{\phi_{q}...\phi_{r}}=\ket{\phi_{p}\phi_{q}...\phi_{r}}
\end{align} 
In the same way, we define annihilation operator: 
\begin{align}
a_{p}\ket{\phi_{p}\phi_{q}...\phi_{r}}=\ket{\phi_{q}...\phi_{r}}
\end{align} 
Slater determinant can be written as a chain of creation operators acting on the true (genuine) vacuum state ($\ket{vac}$):
\begin{align}
a_{p}^{\dagger}a_{q}^{\dagger}...a_{r}^{\dagger}\ket{vac}=\ket{\phi_{p}\phi_{q}...\phi_{r}}.
\end{align} 
Action of the annihilation operator onto vacuum state is:
\begin{align}
a_{p}\ket{vac}=0.
\end{align} 
The creation and annihilation operators obey anti commutation rules which can be derived from:
\begin{align}
a_{q}^{\dagger}a_{p}^{\dagger}\ket{vac}=\ket{\phi_{q}\phi_{p}}=-\ket{\phi_{p}\phi_{q}}=-a_{p}^{\dagger}a_{q}^{\dagger}\ket{vac}
\end{align} 
giving the firs anti commutation relation:
\begin{align}
[a_{q}^{\dagger},a_{p}^{\dagger}]_{+}=0
\end{align} 
Analogous relation can be obtained by taking complex conjugated for of the previous anti commutation relation giving:
\begin{align}
[a_{q},a_{p}]_{+}=0
\end{align} 
The final anti commutation relation may be obtained by considering:
\begin{align}
(a_{p}a_{p}^{\dagger}+a_{p}^{\dagger}a_{p})\ket{\phi_{q}...\phi_{r}}=a_{p}a_{p}^{\dagger}\ket{\phi_{q}...\phi_{r}}=\ket{\phi_{q}...\phi_{r}}
\end{align} 
giving
\begin{align}
a_{p}a_{p}^{\dagger}+a_{p}^{\dagger}a_{p}=1
\end{align} 
and 
\begin{align}
&(a_{p}a_{q}^{\dagger}+a_{q}^{\dagger}a_{p})\ket{\phi_{p}...\phi_{r}}=a_{p}a_{q}^{\dagger}\ket{\phi_{p}...\phi_{r}}+a_{q}^{\dagger}a_{p}\ket{\phi_{p}...\phi_{r}}=\\
&=a_{p}\ket{\phi_{q}\phi_{p}...\phi_{r}}+a_{q}^{\dagger}\ket{...\phi_{r}}=-\ket{\phi_{q}...\phi_{r}}+\ket{\phi_{q}...\phi_{r}}=0
\end{align} 
giving the 
\begin{align}
a_{p}a_{q}^{\dagger}+a_{q}^{\dagger}a_{p}=0
\end{align} 
The two expressions can be written using the one expression as:
\begin{align}
[a_{p},a_{q}^{\dagger}]_{+}=\delta_{pq}
\end{align} 

\subsection{Wick's theorem} 
Contraction for the product of two cration/annihalation operators is defined as:
\begin{align}
\contraction{}{A}{B}{}
AB\equiv AB - \{AB\}
\end{align}
where $\{AB\}$ denotes a normal ordered string. Normal ordered string with the respect to true vacuum is in which all annihilation operators are on the right of all creation operators.
An example of normal ordered string is:
\begin{align}
a_{q}^{\dagger}a_{p}\ket{vac}=0\\
a_{p}\ket{vac}=0
\end{align}  
In order to simplify the notation, we define creation and annihilation operators as:
\begin{align}
a_{p}^{\dagger}=a^{p} \\
a_{p}=a_{p}\\
a_{p}^{\dagger}a_{q}=a_{q}^{p}\\
a_{p}^{\dagger}a_{q}^{\dagger}a_{r}a_{s}=a^{pq}_{sr}
\end{align} 
Up to four contraction can be defined using string of two creation/annihilation operators as:
\begin{align}
\contraction{}{A}{B}{}
a_{p}q_{q}&= a_{p}q_{q} - \{a_{p}q_{q}\}\\
\contraction{}{A}{B}{}
a_{p}^{\dagger}q_{q}^{\dagger}&= a_{p}^{\dagger}q_{q}^{\dagger} - \{a_{p}^{\dagger}q_{q}^{\dagger}\}\\
\contraction{}{A}{B}{}
a_{p}^{\dagger}q_{q}&= a_{p}^{\dagger}q_{q} - \{a_{p}^{\dagger}q_{q}\}\\
\contraction{}{A}{B}{}
a_{p}q_{q}^{\dagger}&= a_{p}q_{q}^{\dagger} - \{a_{p}q_{q}^{\dagger}\}=\delta_{pq}
\end{align} 
or equivalently using tensor notation:
\begin{align}
\contraction{}{A}{B}{}
a_{p}q_{q}&= a_{p}q_{q} - \{a_{p}q_{q}\}\\
\contraction{}{A}{B}{}
a^{p}q^{q}&= a^{p}q^{q} - \{a^{p}q^{q}\}\\
\contraction{}{A}{B}{}
a^{p}q_{q}&= a^{p}q_{q} - \{a^{p}q_{q}\}\\
\contraction{}{A}{B}{}
a_{p}q^{q}&= a_{p}q^{q} - \{a_{p}q^{q}\}=\delta_{p}^{q}
\end{align}
Wick's theorem states that any string of creation/annihilation operators can be expressed as the sum of normal ordered string with sums of single, double and so on contractions expressed by:
\begin{align}
AB...CD = \{AB...CD\}+
\sum_{singles}
\contraction{A}{B}{B}{}
\{AB...CD\}+\sum_{doubles}\{AB...CD\}+...
\end{align}

The formalism that we have introduced for true vacuum can be extended to Fermi vacuum or the particle-hole formalism. Fermi vacuum is Hartree-Fock Slater determinant $\ket{\Phi_{0}}$. Occupied orbitals in $\ket{\Phi_{0}}$ denoted by $i,j,k...$ are called hole states, while unoccupied, $a,b,c...$ are particle states. In the case of HF Slater determinant, we define q-creation operators (creating holes and particles), $a_{i}$ and $a^{a}$ and q-annihilation operators (annihilating hole and particles), $a^{i}$ and $a_{a}$. From such definition, we arrive at two possible contractions:

\begin{align}
a^{i}a_{j}=\delta^{i}_{j}\\
a_{a}a^{b}=\delta^{b}_{a}
\end{align}

\subsection{Normal-ordered electronic Hamiltonian using spin-orbital formalism}
Hamiltonian operator expressed in terms of creation and annihilation operators is of the form:
\begin{align}
\hat{H} = \sum_{pq}\bra{p}\hat{h}\ket{q}a^{\dagger}_{p}a_{q}+\frac{1}{4}\sum_{pqrs}\bra{pq}\ket{rs}a_{p}^{\dagger}a_{q}^{\dagger}a_{s}a_{r}
\end{align}
Tensor notation for the matrix elements and operators:
\begin{align}
\bra{p}\hat{h}\ket{q}=h^{q}_{p}\\
\bra{pq}\frac{1}{r_{12}}\ket{rs}=g^{rs}_{pq}
\end{align}
We can extend the notation on one- and two- electron parts of the Hamiltonian operator as:
\begin{align}
h_{1}&=\sum_{pq}\bra{p}\hat{h}\ket{q}a^{\dagger}_{p}a_{q}=h^{q}_{p}a^{p}_{q}\\
h_{2}&=\sum_{pqrs}\bra{pq}\ket{rs}a_{p}^{\dagger}a_{q}^{\dagger}a_{s}a_{r}=\bar{g}^{rs}_{pq}a^{pq}_{rs}
\end{align}
where $\bar{g}^{rs}_{pq} = g^{rs}_{pq} - g^{rs}_{qr}$. Then Hamiltonian is of the form:
\begin{align}
\hat{H}=h^{q}_{p}a^{p}_{q}+\frac{1}{4}\bar{g}^{rs}_{pq}a^{pq}_{rs}
\end{align}
Such defined Hamiltonian has a general form and we want to define it using the normal ordering convention. One particle part of the Hamiltonian in normal-ordered form is:
\begin{align}
\hat{H}_{1}=h^{q}_{p}a^{p}_{q}=h^{q}_{p}a^{p}_{q}=h^{q}_{p}(\{a^{p}a_{q}\}+\delta^{p}_{q}\delta_{q\in i})=h^{q}_{p}(\tilde{a}^{p}_{q}+\delta^{p}_{q}\delta_{q\in i})=h^{q}_{p}\tilde{a}^{p}_{q}+\sum_{i}^{occ}h^{i}_{i}
\end{align}
while two-particle part of the Hamiltonian operator in the normal ordered form is:
\begin{align}
\hat{H}_{2}&=\frac{1}{4}\bar{g}_{pq}^{rs}a^{pq}_{rs}\\\nonumber
&=\frac{1}{4}\bar{g}_{pq}^{rs}(\tilde{a}^{pq}_{rs})+\tilde{a}^{p}_{r}\delta^{q}_{s}\delta_{s\in i} + \tilde{a}^{q}_{s}\delta^{p}_{r}\delta_{r\in i} - \tilde{a}^{q}_{r}\delta^{p}_{s}\delta_{s\in i} - 
\tilde{a}^{p}_{s}\delta^{q}_{r}\delta_{r\in i} + \delta^{q}_{s}\delta_{s\in i}\delta^{p}_{r}\delta_{r\in j} - \delta^{p}_{s}\delta_{s\in i}\delta^{q}_{r}\delta_{r\in j}\\\nonumber
&=\frac{1}{4}(\bar{g}^{rs}_{pq}\tilde{a}^{pq}_{rs}+\bar{g}^{ri}_{pq}\tilde{a}^{p}_{r}+\bar{g}^{is}_{iq}\tilde{a}^{q}_{s}-\bar{g}^{ri}_{iq}\tilde{a}^{q}_{r}-\bar{g}^{is}_{pi}\tilde{a}^{p}_{s}+\bar{g}^{ji}_{ji}-\bar{g}^{ji}_{ij})=\frac{1}{4}\bar{g}^{rs}_{pq}\tilde{a}^{pq}_{rs}+\bar{g}^{qi}_{pq}\tilde{a}^{p}_{q}+\frac{1}{2}\bar{g}^{ij}_{ij}
\end{align}
Combining these two contributions to form total Hamiltonian, we obtain:
\begin{align}
\hat{H}=h^{i}_{i}+\frac{1}{2}\bar{g}^{ij}_{ij} + (h^{p}_{q}+\bar{g}^{iq}_{ip})\tilde{a}^{p}_{q}+\frac{1}{4}\bar{g}^{rs}_{pq}\tilde{a}^{pq}_{rs}
\end{align}
The three distinct contributions are as follows:
\begin{align}
E_{\text{HF}}&=h^{i}_{i}+\frac{1}{2}\bar{g}^{ij}_{ij}\\
\hat{F}_{N}&=(h^{p}_{q}+\bar{g}^{iq}_{ip})\tilde{a}^{p}_{q}\\
\hat{W}_{N}&=\frac{1}{4}\bar{g}^{rs}_{pq}\tilde{a}^{pq}_{rs}
\end{align}
with $E_{\text{HF}}$ being Hartree-Fock energy, $F_{N}$ normal-ordered Fock operator and $W_{N}$ is fluctuation operator. This allows us to define normal-ordered Hamiltonian as:
\begin{align}
\hat{H}_{N}= \hat{H}-\bra{\Psi_{0}}\hat{H}\ket{\Psi_{0}}=\hat{F}_{N}+\hat{W}_{N}
\end{align}

\subsection{Normal-ordered spin-free Hamiltonian}
One-particle Hamiltonian is:
\begin{align}
\hat{H}_{1}&=h^{p}_{q}a^{q}_{p}=\sum_{pq}\sum_{\rho\sigma}h^{p_{\rho}}_{q_{\sigma}}a^{q_{\sigma}}_{p_{\rho}}=\sum_{pq}\sum_{\rho\sigma}\bra{\phi_{q}\sigma} \hat{h}\ket{\phi_{p}\rho} a^{q_{\sigma}}_{p_{\rho}}\\
&=\sum_{pq}\sum_{\rho\sigma}\bra{\phi_{q}} \hat{h}\ket{\phi_{p}} \braket{\sigma|\rho}a^{q_{\sigma}}_{p_{\rho}}=\sum_{pq}\sum_{\rho=\alpha,\beta}\bra{\phi_{q}} \hat{h}\ket{\phi_{p}}a^{q_{\rho}}_{p_{\rho}}\\
&=\sum_{pq}\bra{\phi_{q}} \hat{h}\ket{\phi_{p}}\sum_{\rho=\alpha,\beta}a^{q_{\rho}}_{p_{\rho}}=\sum_{pq}\bra{\phi_{q}} \hat{h}\ket{\phi_{p}}(a^{q_{\alpha}}_{p_{\alpha}}+a^{q_{\beta}}_{p_{\beta}})
\end{align}
where $\rho,\sigma=\{\alpha,\beta\}$ and $\braket{\sigma|\rho}=\delta^{\sigma}_{\rho}$. We can define spin-free equivalent of the one particle replacer as $E^{q}_{p}=a^{q_{\alpha}}_{p_{\alpha}}+a^{q_{\beta}}_{p_{\beta}}$.

The two-particle part of the Hamiltonian is:
\begin{align}
\hat{H}_{2}&=\frac{1}{2}g^{pq}_{rs}a^{rs}_{pq}=\frac{1}{2}\sum_{pqrs}\sum_{\rho\sigma\mu\nu}\braket{\phi_{p}\rho\phi_{q}\sigma|\phi_{r}\mu\phi_{s}\nu}a^{r_{\mu}s_{\nu}}_{p_{\rho}q_{\sigma}}\\
&=\frac{1}{2}\sum_{pqrs}\sum_{\rho\sigma\mu\nu}\braket{\phi_{p}\phi_{q}|\phi_{r}\phi_{s}}\braket{\rho|\sigma}\braket{\mu|\nu}a^{r_{\mu}s_{\nu}}_{p_{\rho}q_{\sigma}}=\frac{1}{2}\sum_{pqrs}\braket{\phi_{p}\phi_{q}|\phi_{r}\phi_{s}}\sum_{\rho\sigma}a^{r_{\rho}s_{\sigma}}_{p_{\rho}q_{\sigma}}\\
&=\frac{1}{2}\sum_{pqrs}\braket{\phi_{p}\phi_{q}|\phi_{r}\phi_{s}}(a^{r_{\alpha}s_{\alpha}}_{p_{\alpha}q_{\alpha}}+a^{r_{\alpha}s_{\beta}}_{p_{\alpha}q_{\beta}}+a^{r_{\beta}s_{\alpha}}_{p_{\beta}q_{\alpha}}+a^{r_{\beta}s_{\beta}}_{p_{\beta}q_{\beta}})
\end{align}
where we define two-electron replacer $E^{rs}_{pq}=a^{r_{\alpha}s_{\alpha}}_{p_{\alpha}q_{\alpha}}+a^{r_{\alpha}s_{\beta}}_{p_{\alpha}q_{\beta}}+a^{r_{\beta}s_{\alpha}}_{p_{\beta}q_{\alpha}}+a^{r_{\beta}s_{\beta}}_{p_{\beta}q_{\beta}}$. 

Spin-free Hamiltonian is $\hat{H}=h^{p}_{q}E^{q}_{p}+\frac{1}{2}g^{pq}_{rs}E^{rs}_{pq}$. One-particle part of the Hamiltonian is:
\begin{align}
\hat{H}_{1}&=h^{p}_{q}E^{q}_{p}=h^{p}_{q}(\tilde{E}^{q}_{p}+\delta^{q}_{p}\delta_{p \in i}n_{i})=h^{p}_{q}\tilde{E}^{q}_{p}+2\sum_{i}h^{i}_{i}
\end{align}
where $n_{i}$ is number of electrons in occupied orbital i. For closed-shell systems $n_{i}=2$.

The two-particle part of the Hamiltonian:
\begin{align}
\hat{H}_{2}&=\frac{1}{2}g^{pq}_{rs}E^{rs}_{pq}\\\nonumber
&=\frac{1}{2}g^{pq}_{rs}(\tilde{E}^{rs}_{pq}+\tilde{E}^{r}_{p}\delta^{s}_{q}\delta_{q \in i}n_{i})+\tilde{E}^{s}_{q}\delta^{r}_{p}\delta_{p \in i}n_{i}-\tilde{E}^{r}_{q}\delta^{s}_{p}\delta_{p \in i}n_{i}/2-\tilde{E}^{s}_{p}\delta^{r}_{q}\delta_{q \in i}n_{i}/2\\\nonumber
&+\delta^{s}_{q}\delta^{r}_{p}\delta_{q \in i}\delta_{p \in j}n_{i}n_{j}-\delta^{s}_{p}\delta^{r}_{q}\delta_{q \in i}\delta_{p \in j}n_{i}n_{j}/2\\\nonumber
&=\frac{1}{2}g^{pq}_{rs}\tilde{E^{rs}_{pq}}+2g^{ip}_{iq}\tilde{E}^{q}_{p}-g^{ip}_{qi}\tilde{E}^{q}_{p}+2g^{ij}_{ij}-g^{ij}_{ji}
\end{align}

Together, these two contributions to total Hamiltonian in normal-ordered form gives:
\begin{align}
\hat{H}=(h^{p}_{q}+2g^{ip}_{iq}-g^{pi}_{iq})\tilde{E}^{q}_{p}+\frac{1}{2}g^{pq}_{rs}\tilde{E}^{rs}_{pq}+2h^{i}_{i}+2g^{ij}_{ij}-g^{ji}_{ij}
\end{align} 
with $\hat{F}_{N}=(h^{p}_{q}+2g^{ip}_{iq}-g^{pi}_{iq})\tilde{E}^{q}_{p}$, $\hat{W}_{N}=\frac{1}{2}g^{pq}_{rs}\tilde{E}^{rs}_{pq}$ and $E_{\text{HF}}=2h^{i}_{i}+2g^{ij}_{ij}-g^{ji}_{ij}$.

\section{M{\o}ller-Plesset second-order perturbation theory}
In this section we will apply machinery developed in the previous sections for derivation of the M{\o}ller-Plesset second-order perturbation theory equations. We will tackle this problem using two different, by minimizing Hylleraas functional and via first-order wave-equations. We will do that for the spin-orbital and spin-free formalism.

\subsection{Minimization of Hylleraas functional}
Hylleraas second-order functional is defined by:
\begin{align}
H^{(2)}=2\bra{0}\hat{W}_{N}\ket{1}+\bra{1}\hat{F}_{N}\ket{1}
\end{align}
with the $\ket{0}=\ket{\Phi_{0}}=\ket{\text{HF}}$ while the first-order wave-function is defined by: $\ket{1}=\hat{T}\ket{0}=\frac{1}{4}t^{ij}_{ab}\tilde{a}^{ab}_{ij}$.
Rules for generalized Wick's theorem in spin-orbital form are that every hole contraction and every closed loop will give one negative sign. First part of the Hylleraas functional can be resolved as follows:
\begin{align}
\bra{0}\hat{W}_{N}\ket{1}=\bra{0}\frac{1}{4}\bar{g}^{pq}_{rs}\tilde{a}^{rs}_{pq}\frac{1}{4}t^{ij}_{ab}\tilde{a}^{ab}_{ij}\ket{0}=\frac{1}{16}\bar{g}^{pr}_{rs}t^{ij}_{ab}\bra{0}\tilde{a}^{rs}_{pq}\tilde{a}^{ab}_{ij}\ket{0}
\end{align}
Only fully contracted terms will give result different than zero.
\begin{align}
\tilde{a}^{rs}_{pq}\tilde{a}^{ab}_{ij}&=\delta^{r}_{i}\delta^{s}_{j}\delta^{a}_{p}\delta^{b}_{q}\\
\tilde{a}^{rs}_{pq}\tilde{a}^{ab}_{ij}&=-\delta^{r}_{i}\delta^{s}_{j}\delta^{b}_{p}\delta^{a}_{q}\\
\tilde{a}^{rs}_{pq}\tilde{a}^{ab}_{ij}&=-\delta^{s}_{i}\delta^{r}_{j}\delta^{a}_{p}\delta^{b}_{q}\\
\tilde{a}^{rs}_{pq}\tilde{a}^{ab}_{ij}&=\delta^{r}_{i}\delta^{s}_{j}\delta^{a}_{q}\delta^{b}_{p}
\end{align}
giving 
\begin{align}
\bra{0}\hat{W}_{N}\ket{1}=\frac{1}{16}t^{ij}_{ab}(\bar{g}^{ab}_{ij}-\bar{g}^{ba}_{ij}-\bar{g}^{ab}_{ji}+\bar{g}^{ba}_{ji})=\frac{1}{16}t^{ij}_{ab}(\bar{g}^{ab}_{ij}
+\bar{g}^{ab}_{ij}+\bar{g}^{ab}_{ij}+\bar{g}^{ab}_{ij})=\frac{1}{4}t^{ij}_{ab}\bar{g}^{ab}_{ij}
\end{align}
Finally, first part of the Hyllearaas functional contribution will be:
\begin{align}
2\bra{0}\hat{W}_{N}\ket{1}=\frac{1}{2}t^{ij}_{ab}\bar{g}^{ab}_{ij}
\end{align}
The second part of the Hylleraas functional is more involved due to more possible contractions that may occur. We can resolve matrix elements as:
\begin{align}
\bra{1}\hat{F}_{N}\ket{1}=\bra{0}\hat{T}^{\dagger}\hat{F}_{N}\hat{T}\ket{0}=\bra{0}\frac{1}{4}t^{cd}_{kl}\tilde{a}^{kl}_{cd}F^{p}_{q}\tilde{a}^{p}_{q}\frac{1}{4}t^{ij}_{ab}\tilde{a}^{ab}_{ij}\ket{0}=\frac{1}{16}t^{cd}_{kl}F^{p}_{q}t^{ij}_{ab}\bra{0}\tilde{a}^{kl}_{cd}\tilde{a}^{p}_{q}\tilde{a}^{ab}_{ij}\ket{0}
\end{align}
with $\hat{T}=\frac{1}{4}t^{ij}_{ab}\tilde{a}^{ab}_{ij}$, $\hat{T}^{\dagger}=\frac{1}{4}t^{cd}_{kl}\tilde{a}^{kl}_{cd}$ and $\hat{F}_{N}=F^{q}_{p}\tilde{a}^{p}_{q}$. As before, only fully contracted terms will contribute to the final result that may be written as:
\begin{align}
\tilde{a}^{kl}_{cd}\tilde{a}^{p}_{q}\tilde{a}^{ab}_{ij}t^{cd}_{kl}F^{q}_{p}&= +\delta^{p}_{i}\delta^{k}_{j}\delta^{l}_{q}\delta^{a}_{c}\delta^{b}_{d}t^{cd}_{kl}F^{q}_{p}=t^{ab}_{jl}F^{l}_{i}=-t^{ab}_{kj}F^{k}_{i}\\
\tilde{a}^{kl}_{cd}\tilde{a}^{p}_{q}\tilde{a}^{ab}_{ij}t^{cd}_{kl}F^{q}_{p}&= -\delta^{p}_{i}\delta^{k}_{j}\delta^{l}_{q}\delta^{b}_{c}\delta^{a}_{d}t^{cd}_{kl}F^{q}_{p}=-t^{ba}_{jl}F^{l}_{i}=-t^{ab}_{kj}F^{k}_{i}\\
\tilde{a}^{kl}_{cd}\tilde{a}^{p}_{q}\tilde{a}^{ab}_{ij}t^{cd}_{kl}F^{q}_{p}&= -\delta^{p}_{i}\delta^{l}_{j}\delta^{k}_{q}\delta^{a}_{c}\delta^{b}_{d}t^{cd}_{kl}F^{q}_{p}=-t^{ab}_{kj}F^{k}_{i}=-t^{ab}_{kj}F^{k}_{i}\\
\tilde{a}^{kl}_{cd}\tilde{a}^{p}_{q}\tilde{a}^{ab}_{ij}t^{cd}_{kl}F^{q}_{p}&= +\delta^{p}_{i}\delta^{l}_{j}\delta^{k}_{q}\delta^{b}_{c}\delta^{a}_{d}t^{cd}_{kl}F^{q}_{p}=t^{ba}_{kj}F^{k}_{i}=-t^{ab}_{kj}F^{k}_{i}\\
\tilde{a}^{kl}_{cd}\tilde{a}^{p}_{q}\tilde{a}^{ab}_{ij}t^{cd}_{kl}F^{q}_{p}&= -\delta^{p}_{j}\delta^{k}_{i}\delta^{l}_{q}\delta^{a}_{c}\delta^{b}_{d}t^{cd}_{kl}F^{q}_{p}=-t^{ab}_{il}F^{l}_{j}=-t^{ab}_{ik}F^{k}_{j}\\
\tilde{a}^{kl}_{cd}\tilde{a}^{p}_{q}\tilde{a}^{ab}_{ij}t^{cd}_{kl}F^{q}_{p}&=+ \delta^{p}_{j}\delta^{k}_{i}\delta^{l}_{q}\delta^{a}_{d}\delta^{b}_{c}t^{cd}_{kl}F^{q}_{p}=t^{ba}_{il}F^{l}_{j}=-t^{ab}_{ik}F^{k}_{j}\\
\tilde{a}^{kl}_{cd}\tilde{a}^{p}_{q}\tilde{a}^{ab}_{ij}t^{cd}_{kl}F^{q}_{p}&= \delta^{p}_{j}\delta^{l}_{i}\delta^{k}_{q}\delta^{a}_{c}\delta^{b}_{d}t^{cd}_{kl}F^{q}_{p}=-t^{ab}_{ki}F^{k}_{j}=-t^{ab}_{ik}F^{k}_{j}\\
\tilde{a}^{kl}_{cd}\tilde{a}^{p}_{q}\tilde{a}^{ab}_{ij}t^{cd}_{kl}F^{q}_{p}&=- \delta^{p}_{j}\delta^{l}_{i}\delta^{k}_{q}\delta^{a}_{d}\delta^{b}_{c}t^{cd}_{kl}F^{q}_{p}=-t^{ba}_{ki}F^{k}_{j}=-t^{ab}_{ik}F^{k}_{j}\\
\tilde{a}^{kl}_{cd}\tilde{a}^{p}_{q}\tilde{a}^{ab}_{ij}t^{cd}_{kl}F^{q}_{p}&=- \delta^{p}_{c}\delta^{a}_{d}\delta^{b}_{q}\delta^{k}_{i}\delta^{l}_{j}t^{cd}_{kl}F^{q}_{p}=-t^{ca}_{ij}F^{b}_{c}=t^{ac}_{ij}F^{b}_{c}\\
\tilde{a}^{kl}_{cd}\tilde{a}^{p}_{q}\tilde{a}^{ab}_{ij}t^{cd}_{kl}F^{q}_{p}&=+ \delta^{p}_{c}\delta^{a}_{d}\delta^{b}_{q}\delta^{l}_{i}\delta^{k}_{j}t^{cd}_{kl}F^{q}_{p}=t^{ca}_{ji}F^{b}_{c}=t^{cb}_{ij}F^{a}_{c}\\
\tilde{a}^{kl}_{cd}\tilde{a}^{p}_{q}\tilde{a}^{ab}_{ij}t^{cd}_{kl}F^{q}_{p}&=+ \delta^{p}_{c}\delta^{b}_{d}\delta^{a}_{q}\delta^{k}_{i}\delta^{l}_{j}t^{cd}_{kl}F^{q}_{p}=t^{cb}_{ij}F^{a}_{c}=t^{cb}_{ij}F^{a}_{c}\\
\tilde{a}^{kl}_{cd}\tilde{a}^{p}_{q}\tilde{a}^{ab}_{ij}t^{cd}_{kl}F^{q}_{p}&=- \delta^{p}_{c}\delta^{b}_{d}\delta^{a}_{q}\delta^{l}_{i}\delta^{k}_{j}t^{cd}_{kl}F^{q}_{p}=-t^{cb}_{ji}F^{a}_{c}=t^{cb}_{ij}F^{a}_{c}\\
\tilde{a}^{kl}_{cd}\tilde{a}^{p}_{q}\tilde{a}^{ab}_{ij}t^{cd}_{kl}F^{q}_{p}&=+ \delta^{p}_{d}\delta^{a}_{c}\delta^{b}_{q}\delta^{k}_{i}\delta^{l}_{j}t^{cd}_{kl}F^{q}_{p}=t^{ad}_{ij}F^{b}_{d}=t^{ac}_{ij}F^{b}_{c}\\
\tilde{a}^{kl}_{cd}\tilde{a}^{p}_{q}\tilde{a}^{ab}_{ij}t^{cd}_{kl}F^{q}_{p}&=- \delta^{p}_{d}\delta^{a}_{c}\delta^{b}_{q}\delta^{k}_{j}\delta^{l}_{i}t^{cd}_{kl}F^{q}_{p}=-t^{ad}_{ji}F^{b}_{d}=t^{ac}_{ij}F^{b}_{c}\\
\tilde{a}^{kl}_{cd}\tilde{a}^{p}_{q}\tilde{a}^{ab}_{ij}t^{cd}_{kl}F^{q}_{p}&=- \delta^{p}_{d}\delta^{b}_{c}\delta^{b}_{q}\delta^{k}_{i}\delta^{l}_{j}t^{cd}_{kl}F^{q}_{p}=-t^{bd}_{ij}F^{a}_{d}=t^{cb}_{ij}F^{a}_{c}\\
\tilde{a}^{kl}_{cd}\tilde{a}^{p}_{q}\tilde{a}^{ab}_{ij}t^{cd}_{kl}F^{q}_{p}&=- \delta^{p}_{d}\delta^{b}_{c}\delta^{a}_{q}\delta^{k}_{j}\delta^{l}_{i}t^{cd}_{kl}F^{q}_{p}=t^{bd}_{ji}F^{a}_{d}=t^{cb}_{ij}F^{a}_{c}
\end{align}
giving the
\begin{align}
\bra{1}\hat{F}_{N}\ket{1}=\frac{1}{4}t^{ij}_{ab}(t^{ac}_{ij}F^{b}_{c}+t^{cb}_{ij}F^{a}_{c}-t^{ab}_{kj}F^{ki}-t^{ab}_{ik}F^{k}_{j})
\end{align}
The final for of the second-order Hylleraas functional is now:
\begin{align}
H^{(0)}(\ket{1})=\frac{1}{4}t^{ij}_{ab}(2\bar{g}^{ab}_{ij}+t^{ac}_{ij}F^{b}_{c}+t^{cb}_{ij}F^{a}_{c}-t^{ab}_{kj}F^{ki}-t^{ab}_{ik}F^{k}_{j})
\end{align}
To find optimal amplitudes, we need to minimize the Hylleraas functional as:
\begin{align}
\frac{\partial H^{(2)}}{\partial t^{mn}_{ef}}=0
\end{align}

Thus the first term will be of the form:
\begin{align}
\frac{\partial }{\partial t^{mn}_{ef}}\bigg(\frac{1}{2}t^{ij}_{ab}\bar{g}^{ab}_{ij}\bigg)=\frac{1}{2}\delta^{i}_{m}\delta^{j}_{n}\delta^{a}_{l}\delta^{b}_{f}\bar{g}^{ab}_{ij}=\frac{1}{2}\bar{g}^{ef}_{mn}=\frac{1}{2}\bar{g}^{ab}_{ij}
\end{align}
The second term will be:
\begin{align}
\frac{\partial }{\partial t^{mn}_{ef}}\bigg(\frac{1}{4}t^{ij}_{ab}t^{ac}_{ij}F^{b}_{c}\bigg)&=\frac{1}{4}\bigg(\frac{\partial t^{ij}_{ab}}{\partial t^{mn}_{ef}}t^{ac}_{ij}+t^{ij}_{ab}\frac{\partial t^{ac}_{ij}}{\partial t^{mn}_{ef}}\bigg)F^{b}_{c}=\frac{1}{4}F^{b}_{c}(\delta^{i}_{m}\delta^{j}_{n}\delta^{a}_{l}\delta^{b}_{f}t^{ac}_{ij}+t^{ij}_{ab}\delta^{a}_{e}\delta^{c}_{f}\delta^{i}_{m}\delta^{j}_{n})\\\nonumber
&=\frac{1}{4}(t^{ec}_{mn}F^{f}_{c}+t^{mn}_{eb}F^{b}_{f})=\frac{1}{2}t^{ec}_{mn}F^{f}_{c}=\frac{1}{2}t^{ac}_{ij}F^{b}_{c}
\end{align}
The remaining terms are resolved in the similar manner as the previous one.
Finally, we can write amplitude equation of also known as MP1 equations as:
\begin{align}
\frac{\partial H^{(2)}}{\partial t}=\frac{1}{2}\bar{g}^{ab}_{ij}+\frac{1}{2}t^{cb}_{ij}F^{a}_{c}+\frac{1}{2}t^{ac}_{ij}F^{b}_{c}-\frac{1}{2}t^{ab}_{kj}F^{i}_{k}-\frac{1}{2}t^{ab}_{ik}F^{j}_{k}=0
\end{align}
as:
\begin{align}
R^{ab}_{ij}=\bar{g}^{ab}_{ij}+t^{cb}_{ij}F^{a}_{c}+t^{ac}_{ij}F^{b}_{c}-t^{ab}_{kj}F^{i}_{k}-t^{ab}_{ik}F^{j}_{k}=0
\end{align}

\subsection{R-S perturbation theory}
In order to derive the amplitude equation, we can start from the MP1 equations that we have derived in the context of the R-S perturbation theory:
\begin{align}
\hat{H}^{(1)}\ket{0}+\hat{H}^{(0)}\ket{1}=E^{(1)}\ket{0}+E^{(0)}\ket{1}
\end{align}
with $\hat{H}=\hat{H}^{(0)}+\hat{H}^{(1)}=E_{\text{HF}}+\hat{F}_{N}+\hat{W}_{N}$ where we define $\hat{H}^{(0)}=E_{\text{HF}}+\hat{F}_{N}$ and $\hat{H}^{(1)}=\hat{W}_{N}$. Now MP1 equations reads as:
\begin{align}
\hat{W}_{N}\ket{0}+(E_{\text{HF}}+\hat{F}_{N})\ket{1}=E^{(1)}\ket{0}+E^{(0)}\ket{1}
\end{align} 
which can be solved by projection on the right hand side onto the space of doubly excited determinants giving:
\begin{align}
\bra{^{kl}_{cd}}\hat{W}_{N}\ket{0}+\bra{^{kl}_{cd}}\hat{F}_{N}\ket{1}=0
\end{align}   
By resolving the matrix elements using the Wick's theorem rules, we arrive at the amplitude equation expression:
\begin{align}
R^{ab}_{ij}=\bar{g}^{ab}_{ij}+t^{cb}_{ij}F^{a}_{c}+t^{ac}_{ij}F^{b}_{c}-t^{ab}_{kj}F^{i}_{k}-t^{ab}_{ik}F^{j}_{k}=0
\end{align}

\subsection{Minimization of Hylleraas functional in the spin-free form}
We star the derivation of the MP1 amplitude equations from the second-order Hylleraas functional
\begin{align}
H^{(0)}=2\bra{0}\hat{W}_{N}\ket{1}+\bra{1}\hat{F}_{N}\ket{1}
\end{align}
where $\ket{1}=\frac{1}{2}t^{ij}_{ab}E^{ab}_{ij}$, $\hat{W}_{N}=]\frac{1}{2}g^{pq}_{rs}\tilde{E^{rs}_{pq}}$ and $\hat{F}_{N}=F^{p}_{q}\tilde{E}^{q}_{p}$ where $\ket{1}$ is already in normal-ordered form. The rules for Wick's theorem in the spin-free formalism are that every hole contraction contributes with one minus sign and every closed loop contribute with -2. Having that in mind, we can resolve the matrix elements of the Hylleraas functional staring from the first contribution:

\begin{align}
\bra{0}\hat{W}_{N}\ket{1}=\bra{0}\frac{1}{2}g^{pq}_{rs}\tilde{E}^{rs}_{pq}\frac{1}{2}t^{ij}_{ab}E^{ab}_{ij}\ket{0}=\frac{1}{4}g^{pq}_{rs}t^{ij}_{ab}\bra{0}\tilde{E}^{rs}_{pq}E^{ab}_{ij}\ket{0}
\end{align}

then the elements inside brackets can be resolved as:
\begin{align}
\tilde{E^{rs}_{pq}}\tilde{E}^{ab}_{ij}&=+4\delta^{r}_{i}\delta^{s}_{j}\delta^{a}_{p}\delta^{b}_{q}\\
\tilde{E^{rs}_{pq}}\tilde{E}^{ab}_{ij}&=-2\delta^{r}_{i}\delta^{s}_{j}\delta^{b}_{p}\delta^{a}_{q}\\
\tilde{E^{rs}_{pq}}\tilde{E}^{ab}_{ij}&=-2\delta^{s}_{i}\delta^{r}_{j}\delta^{a}_{p}\delta^{b}_{q}\\
\tilde{E^{rs}_{pq}}\tilde{E}^{ab}_{ij}&=+4\delta^{s}_{i}\delta^{r}_{j}\delta^{b}_{p}\delta^{a}_{q}
\end{align}
Inserting these 4 expressions, we obtain:


\begin{align}
\bra{0}\hat{W}_{N}\ket{1}=\frac{1}{4}(8g^{ab}_{ij}t^{ij}_{ab}-4g^{ba}_{ij}t^{ij}_{ab})=(2g^{ab}_{ij}-g^{ba}_{ij})t^{ij}_{ab}=\bar{g}^{ab}_{ij}t^{ij}_{ab}
\end{align}
where $\bar{g}^{ab}_{ij}=2g^{ab}_{ij}-g^{ba}_{ij}$ antisymmertized two-electron integrals equivalent in the spin-free formalism. We used the same notation for the spin-orbital formalism but now context is different. From now on, we will only use spin-free formalism in derivation of the F12 methods.

The resolution of the second part of the Hylleraas functional $\bra{1}\hat{F}_{N}\ket{1}$ is:
\begin{align}
\bra{1}\hat{F}_{N}\ket{1}=\bra{0}\hat{T}^{\dagger}\hat{F}_{N}\hat{T}\ket{0}=\frac{1}{4}t^{cd}_{kl}F^{p}_{q}t^{ij}_{ab}\bra{0}\tilde{E}^{kl}_{cd}\tilde{F}^{q}_{p}\tilde{E}^{ab}_{ij}\ket{0}
\end{align}
The expression inside the brackets can be resolved as:
\begin{align}
\tilde{E}^{kl}_{cd}\tilde{F}^{q}_{p}\tilde{E}^{ab}_{ij}=-4\delta^{q}_{i}\delta^{l}_{j}\delta^{k}_{p}\delta^{a}_{c}\delta^{b}_{d}\\
\tilde{E}^{kl}_{cd}\tilde{F}^{q}_{p}\tilde{E}^{ab}_{ij}=+2\delta^{q}_{i}\delta^{l}_{j}\delta^{k}_{p}\delta^{b}_{c}\delta^{a}_{d}
\end{align}
with 16 elements in total. After resolving matrix elements, we arrive at the expression for the Hylleraas expression:
\begin{align}
H^{(0)}&=2\bra{0}\hat{W}_{N}\ket{1}+\bra{1}\hat{F}_{N}\ket{1}=(4g^{ab}_{ij}-2g^{ba}_{ij})t^{ij}_{ab}\\\nonumber
&+2t^{ij}_{ab}t^{ac}_{ij}F^{b}_{c}-t^{ij}_{ab}t^{ca}_{ij}F^{b}_{c}+2t^{ij}_{ab}t^{cb}_{ij}F^{a}_{c}-t^{ij}_{ab}t^{bc}_{ij}F^{a}_{c}-2t^{ij}_{ab}t^{ab}_{kj}F^{k}_{i}+t^{ij}_{ab}t^{ba}_{kj}F^{k}_{i}-2t^{ij}_{ab}t^{ab}_{ik}F^{k}_{j}+t^{ij}_{ab}t^{ba}_{ik}F^{k}_{j}
\end{align}
Minimization of the Hyllearaas function with the respect to the amplitudes, as was derived in the previous section gives the set of MP1 amplitude equations in spin-free formalism:
\begin{align}
R^{ij}_{ab}=\bar{g}^{ij}_{ab}
+\bar{t}^{ij}_{ac}F^{c}_{b}+\bar{t}^{ij}_{cb}F^{c}_{a}-\bar{t}^{kj}_{ab}F^{i}_{k}-\bar{t}^{ik}_{ab}F^{j}_{k}
\end{align}
with $\bar{t}^{ij}_{ab}=2t^{ij}_{ab}-t^{ij}_{ba}$. By solving the linear system of equations, we obtain optimized amplitudes that can be used for evaluation of te second order energy contribution also know as MP2 energy as:

\begin{align}
E^{(2)}=\bra{0}\hat{H}^{(1)}\ket{1}=\bra{0}\hat{W}_{N}\ket{1}=\bar{g}^{ab}_{ij}t^{ij}_{ab}=g^{ab}_{ij}\bar{t}^{ij}_{ab}
\end{align} 

Selecting the canonical orbitals, in which Fock matrix has diagonal form, we arrive as the well known expression for the MP2 energy correction as:

\begin{align}
E^{(2)}=\sum_{ijab}\frac{(2g^{ab}_{ij}-g^{ba}_{ij})t^{ij}_{ab}}{\epsilon_{i}+\epsilon_{j}-\epsilon_{a}-\epsilon_{b}}
\end{align} 

\section{Explicitly correlated methods}

\subsection{MP2-F12 method}
Within explicitly correlated formalism, the first-order wave function in the spin-free formalism is
\begin{align}
\ket{1}=\ket{1_{\text{MP2}}}+\ket{1_{\text{F12}}}=\frac{1}{2}t^{ij}_{ab}\tilde{E}^{ab}_{ij}\ket{0}+\frac{1}{2}\tilde{R}^{ij}_{\alpha\beta}\tilde{E}^{\alpha\beta}_{ij}\ket{0}
\end{align} 
where $\tilde{R}^{ij}_{\alpha\beta}$ is matrix element of the explicit-correlation factor.

The Hylleraas functional for the F12 augmented MP1 wave function is
\begin{align}
H^{(2)}_{\text{MP2-F12}}&=\bra{1}\hat{F}_{N}\ket{1}+2\bra{0}\hat{W}_{N}\ket{1}\\
&=\bra{1_{\text{MP2}}}\hat{F}_{N}\ket{1_{\text{MP2}}}+2\bra{0}\hat{W}_{N}\ket{1_{\text{MP2}}}+\bra{1_{\text{MP2}}}\hat{F}_{N}\ket{1_{\text{F12}}}\\
&+\bra{1_{\text{F12}}}\hat{F}_{N}\ket{1_{\text{MP2}}}+\bra{1_{\text{F12}}}\hat{F}_{N}\ket{1_{\text{F12}}}+2\bra{0}\hat{W}_{N}\ket{1_{\text{F12}}}
\end{align} 

having the $H^{(2)}_{\text{MP2}}=\bra{1_{\text{MP2}}}\hat{F}_{N}\ket{1_{\text{MP2}}}+2\bra{0}\hat{W}_{N}\ket{1_{\text{MP2}}}$ and using the adjoint property of the matrix elements, we have:

\begin{align}
H^{(2)}_{\text{MP2-F12}}&=H^{(2)}_{\text{MP2}}+2\bra{1_{\text{MP2}}}\hat{F}_{N}\ket{1_{\text{F12}}}+\bra{1_{\text{F12}}}\hat{F}_{N}\ket{1_{\text{F12}}}+2\bra{0}\hat{W}_{N}\ket{1_{\text{F12}}}
\end{align} 

As we have showed in the previous chapter, the Hylleraas functional for the MP1 wave function in the spin-free formalism is $2\bar{t}^{ij}_{ab}g^{ab}_{ij}
+\bar{t}^{ij}_{ab}t_{ij}^{ac}F_{c}^{b}+\bar{t}^{ij}_{ab}t_{ij}^{cb}F_{c}
^{a}-\bar{t}^{ij}_{ab}t_{kj}^{ab}F_{i}^{k}-\bar{t}^{ij}_{ab}t_{ik}^{ab}F_{j}^{k}$. The only thing that remains is to resolve the following three matrix elements $\bra{1_{\text{MP2}}}\hat{F}_{N}\ket{1_{\text{F12}}}$, $\bra{1_{\text{F12}}}\hat{F}_{N}\ket{1_{\text{F12}}}$ and $\bra{0}\hat{W}_{N}\ket{1_{\text{F12}}}$. The resolution of the $\bra{0}\hat{W}_{N}\ket{1_{\text{F12}}}$ is as follows:

\begin{align}
\bra{0}\hat{W}_{N}\ket{1_{\text{F12}}}=\frac{1}{4}g^{pq}_{rs}\tilde{R}^{ij}_{\alpha\beta}\bra{0}\tilde{E}^{rs}_{pq}\tilde{E}^{\alpha\beta}_{ij}\ket{0}=(2g^{\alpha\beta}_{ij}-g^{\alpha\beta}_{ji})\tilde{R}^{ij}_{\alpha\beta}=2V^{ij}_{ij}-V^{ji}_{ij}=\bar{V}^{ij}_{ij}
\end{align}

where $V^{ij}_{ij}=g^{\alpha\beta}_{ij}\tilde{R}^{ij}_{\alpha\beta}$. Here, the summation over the $\alpha$ and $\beta$ indices is assumed.

The term $\bra{1_{\text{MP2}}}\hat{F}_{N}\ket{1_{\text{F12}}}$ is evaluated as follows:

\begin{align}
\bra{1_{\text{MP2}}}\hat{F}_{N}\ket{1_{\text{F12}}}&=\frac{1}{4}t^{ab}_{ij}F^{p}_{q}\tilde{R}^{kl}_{\alpha\beta}\bra{0}\tilde{E}^{ij}_{ab}\tilde{E}^{p}_{q}\tilde{E}^{\alpha\beta}_{kl}\ket{0}
\end{align}
resolution of the matrix elements will give:

\begin{align}
\tilde{E}^{ij}_{ab}\tilde{E}^{p}_{q}\tilde{E}^{\alpha\beta}_{kl}=-2\delta^{i}_{k}\delta^{j}_{l}\delta^{q}_{a}\delta^{b}_{\alpha}\delta^{\beta}_{p}=-2F^{\beta}_{a}\tilde{R}^{ij}_{b\beta}=-2F^{\alpha}_{a}\tilde{R}^{ij}_{b\alpha}\\
\tilde{E}^{ij}_{ab}\tilde{E}^{p}_{q}\tilde{E}^{\alpha\beta}_{kl}=+4\delta^{i}_{k}\delta^{j}_{l}\delta^{q}_{a}\delta^{b}_{\beta}\delta^{\alpha}_{p}=+4F^{\alpha}_{a}\tilde{R}^{ij}_{\alpha b}=+4F^{\alpha}_{a}\tilde{R}^{ij}_{\alpha b}\\
\end{align}
There are 8 such contractions. Collecting all of them will give:

\begin{align}
2\bra{1_{\text{MP2}}}\hat{F}_{N}\ket{1_{\text{F12}}}&=2t^{ab}_{ij}[F^{\alpha}_{a}(2\tilde{R}^{ij}_{\alpha b}-\tilde{R}^{ji}_{\alpha b})+F^{\alpha}_{b}(2\tilde{R}^{ij}_{a\alpha}-\tilde{R}^{ji}_{a\alpha})]\\
&=2\bar{t}^{ab}_{ij}(F^{\alpha}_{a}\tilde{R}^{ij}_{\alpha b}+F^{\alpha}_{b}\tilde{R}^{ij}_{a\alpha})=2\bar{t}^{ab}_{ij}C^{ij}_{ab}
\end{align}
where $\bar{t}^{ab}_{ij}=2t^{ab}_{ij}-t^{ab}_{ji}$ and intermediate $C^{ij}_{ab}=F^{\alpha}_{a}\tilde{R}^{ij}_{\alpha b}+F^{\alpha}_{b}\tilde{R}^{ij}_{a\alpha}$.

Lastly, we resolve $\bra{1_{\text{F12}}}\hat{F}_{N}\ket{1_{\text{F12}}}$ matrix elements. 

\begin{align}
\bra{1_{\text{F12}}}\hat{F}_{N}\ket{1_{\text{F12}}}=\frac{1}{4}\tilde{R}_{ij}^{\alpha\beta}F^{p}_{q}\tilde{R}^{kl}_{\gamma\delta}\bra{0}\tilde{E}^{ij}_{\alpha\beta}\tilde{E}^{q}_{p}\tilde{E}^{\gamma\delta}_{kl}\ket{0}
\end{align}
Furthermore, the expression in brackets will give two intermediates that arise by 2-hole-3-particle and 3-hole-2-particle contractions. Let us first resolve 2-hole-3-particle. This will give 8 different contractions in total, however, we show here only first few:

\begin{align}
\tilde{E}^{ij}_{\alpha\beta}\tilde{E}^{q}_{p}\tilde{E}^{\gamma\delta}_{kl}=-2\delta^{i}_{k}\delta^{j}_{l}\delta^{q}_{\alpha}\delta^{\gamma}_{\beta}\delta^{\delta}_{p}=-2\tilde{R}^{\alpha\beta}_{ij}F^{\delta}_{\alpha}\tilde{R}^{ij}_{\beta\delta}=-2\tilde{R}^{\alpha\beta}_{ij}F^{\gamma}_{\alpha}\tilde{R}^{ij}_{\beta\gamma}\\
\tilde{E}^{ij}_{\alpha\beta}\tilde{E}^{q}_{p}\tilde{E}^{\gamma\delta}_{kl}=+4\delta^{i}_{k}\delta^{j}_{l}\delta^{q}_{\alpha}\delta^{\delta}_{\beta}\delta^{\gamma}_{p}=+4\tilde{R}^{\alpha\beta}_{ij}F^{\gamma}_{\alpha}\tilde{R}^{ij}_{\gamma\beta}=+4\tilde{R}^{\alpha\beta}_{ij}F^{\gamma}_{\alpha}\tilde{R}^{ij}_{\gamma\beta}
\end{align}  

Collecting all 8 contributions yields:
\begin{align}
\bra{1_{\text{F12}}}\hat{F}_{N}\ket{1_{\text{F12}}}\leftarrow2\tilde{R}^{\alpha\beta}_{ij}F^{\gamma}_{\alpha}\tilde{R}^{ij}_{\gamma\beta}+2\tilde{R}^{\alpha\beta}_{ij}F^{\gamma}_{\beta}\tilde{R}^{ij}_{\alpha\gamma}-\tilde{R}^{\alpha\beta}_{ij}F^{\gamma}_{\alpha}\tilde{R}^{ji}_{\gamma\beta}-\tilde{R}^{\alpha\beta}_{ij}F^{\gamma}_{\beta}\tilde{R}^{ji}_{\alpha\gamma}
\end{align}

we can define a new intermediate $B^{ij}_{ij}=\tilde{R}^{\alpha\beta}_{ij}F^{\gamma}_{\alpha}\tilde{R}^{ij}_{\gamma\beta}+\tilde{R}^{\alpha\beta}_{ij}F^{\gamma}_{\beta}\tilde{R}^{ij}_{\alpha\gamma}$ gives $\bra{1_{\text{F12}}}\hat{F}_{N}\ket{1_{\text{F12}}}\leftarrow\bar{B}^{ij}_{ij}$ where $\bar{B}^{ij}_{ij}=2B^{ij}_{ij}-B^{ji}_{ij}$. Finally, we can resolve the remaining 8 contributions that arise from 3-hole-2-particle contractions. Here, we will skip derivation of the all 8 contractions and proceed to the final result  

\begin{align}
\bra{1_{\text{F12}}}\hat{F}_{N}\ket{1_{\text{F12}}}&\leftarrow-[2\tilde{R}^{\alpha\beta}_{ij}\tilde{R}^{kj}_{\alpha\beta}F^{i}_{k}-\tilde{R}^{\alpha\beta}_{ij}\tilde{R}^{jk}_{\alpha\beta}F^{i}_{k}+2\tilde{R}^{\alpha\beta}_{ij}\tilde{R}^{ik}_{\alpha\beta}F^{j}_{k}-\tilde{R}^{\alpha\beta}_{ij}\tilde{R}^{ki}_{\alpha\beta}F^{j}_{k}]\\\nonumber
&=-[(2X^{kj}_{ij}-X^{jk}_{ij})F^{i}_{k}+(2X^{ik}_{ij}-X^{ki}_{ij})F^{j}_{k}]=-[\tilde{X}^{kj}_{ij}F^{i}_{k}+\tilde{X}^{ik}_{ij}F^{j}_{k}]
\end{align}

where matrix elements of intermediate $X$ are defined as $X^{kj}_{ij}=\tilde{R}^{\alpha\beta}_{ij}\tilde{R}^{kj}_{\alpha\beta}$ and $\bar{X}^{kj}_{ij}=2X^{kj}_{ij}-X^{jk}_{ij}$ are antisymmetrized matrix elements in spin-free formalism. 

Finally, we can collect all the contributions and write final expression for the second order Hylleraas functional:
\begin{align}
H^{(2)}_{\text{MP2-F12}}=&\bra{1}\hat{F}_{N}\ket{1}+2\bra{0}\hat{W}_{N}\ket{1}\\
=&\bra{1_{\text{MP2}}}\hat{F}_{N}\ket{1_{\text{MP2}}}+2\bra{0}\hat{W}_{N}\ket{1_{\text{MP2}}}+\bra{1_{\text{MP2}}}\hat{F}_{N}\ket{1_{\text{F12}}}\\
&+\bra{1_{\text{F12}}}\hat{F}_{N}\ket{1_{\text{MP2}}}+\bra{1_{\text{F12}}}\hat{F}_{N}\ket{1_{\text{F12}}}+2\bra{0}\hat{W}_{N}\ket{1_{\text{F12}}}\\
=&\bar{t}^{ij}_{ab}(g^{ab}_{ij}+C^{ab}_{ij}+\rho^{ab}_{ij})+2\bar{V}^{ij}_{ij}+\bar{B}^{ij}_{ij}-\bar{X}^{kj}_{ij}F^{i}_{k}-\bar{X}^{ik}_{ij}F^{j}_{k}
\end{align} 
where $\rho^{ab}_{ij}=\bra{^{ij}_{ab}}\hat{W}_{N}\ket{0}+\bra{^{ij}_{ab}}\hat{F}_{N}\ket{1_{\text{MP2}}}+\bra{^{ij}_{ab}}\hat{F}_{N}\ket{1_{\text{F12}}}$. For optimized amplitudes, the expression $\bar{t}^{ij}_{ab}\rho^{ab}_{ij}$ gives 0 and for such amplitudes $\text{MP2-F12}$ energy expression is:

\begin{align}
E_{\text{MP2-F12}}=\bar{t}^{ij}_{ab}(g^{ab}_{ij}+C^{ab}_{ij})+2\bar{V}^{ij}_{ij}+\bar{B}^{ij}_{ij}-\bar{X}^{kj}_{ij}F^{i}_{k}-\bar{X}^{ik}_{ij}F^{j}_{k}
\end{align} 
 
\subsection{Perturbative F12 correction to the CCSD energy} 
The energy correction of the perturbative F12 approach is obtained by evaluating the following Hylleraas functional:

\begin{align}
E_{(2)_{\overline{\text{F12}}}}=\bra{1_{\text{F12}}}\hat{H}^{(0)}\ket{1_{\text{F12}}}+\bra{0}\hat{H}^{(1)}\ket{1_{\text{F12}}}+\bra{1_{\text{F12}}}\hat{H}^{(1)}\ket{0}
\end{align} 
 with $\ket{0}=\ket{\Psi_{0}}=\ket{\text{HF}}$ and $\bra{0}=\bra{\Psi_{0}}(1+\hat{T}^{\dagger})$.
Zeroth-order Hamiltonian we take normal-ordered Fock operator while first-order Hamiltonian is normal-ordered similarity transformed Hamiltonian $\bar{H}_{N}$. This, second and third therms would yield many contributions, however, we will only consider contributions that are liner in amplitude expansion of the normal-ordered similarity transformed Hamiltonian. Thus, the first term will give familiar expression that we have derived in the previous section. 

\begin{align}
\mathcal{B}^{ij}_{ij}=\bra{1_{\text{F12}}}\hat{F}_{N}\ket{1_{\text{F12}}}
=\bar{B}^{ij}_{ij}-\bar{X}^{kj}_{ij}F^{i}_{k}-\bar{X}^{ik}_{ij}F^{j}_{k}
\end{align}  

Let us define linear contributions to the normal-ordered similarity-transformed Hamiltonian:
\begin{align}
\bar{H}_{N}=\hat{F}_{N}+\hat{W}_{N}+\hat{F}_{N}\hat{T}_{1}+\hat{W}_{N}\hat{T}_{1}+\hat{F}_{N}\hat{T}_{2}+\hat{W}_{N}\hat{T}_{2}
\end{align}  

Resolution of the third term will give:
\begin{align}
\bra{1_{\text{F12}}}\bar{H}_{N}\ket{0}
=&\bra{1_{\text{F12}}}\hat{F}_{N}\ket{0}+\bra{1_{\text{F12}}}\hat{W}_{N}\ket{0}+\bra{1_{\text{F12}}}\hat{F}_{N}\hat{T}_{1}\ket{0}+\bra{1_{\text{F12}}}\hat{W}_{N}\hat{T}_{1}\ket{0}\\\nonumber
&+\bra{1_{\text{F12}}}\hat{F}_{N}\hat{T}_{2}\ket{0}+\bra{1_{\text{F12}}}\hat{W}_{N}\hat{T}_{2}\ket{0}
\end{align}  

Now, let us analyze each contribution that may arise from this term. The contribution $\bra{1_{\text{F12}}}\hat{F}_{N}\ket{0}$ will give 0 since no fully contracted contribution can be constructed. Second contribution, $\bra{1_{\text{F12}}}\hat{W}_{N}\ket{0}$, is familiar $\bar{V}^{ij}_{ij}$. The third term, $\bra{1_{\text{F12}}}\hat{F}_{N}\hat{T}_{1}\ket{0}$, is also zero due to Generalized-Brillouin condition (GBC) that states that $F^{i}_{\alpha}=0$. The contribution  $\bra{1_{\text{F12}}}\hat{W}_{N}\hat{T}_{1}\ket{0}$ will yield $\bar{V}^{ia}_{ij}t^{j}_{a}+\bar{V}^{aj}_{ij}t^{i}_{a}$. The contribution $\bra{1_{\text{F12}}}\hat{F}_{N}\hat{T}_{2}\ket{0}$ is familiar $C$ intermediate contracted with double amplitudes $\bar{C}^{ab}_{ij}t^{ij}_{ab}$. Lastly, $\bra{1_{\text{F12}}}\hat{W}_{N}\hat{T}_{2}\ket{0}$ will give $\bar{V}^{ab}_{ij}t^{ij}_{ab}$. Finally, we can write whole expression as:
\begin{align}
\bra{1_{\text{F12}}}\bar{H}_{N}\ket{0}
=\bar{V}^{ij}_{ij}+\bar{V}^{ia}_{ij}t^{j}_{a}+\bar{V}^{aj}_{ij}t^{i}_{a}+\bar{C}^{ab}_{ij}t^{ij}_{ab}+\bar{V}^{ab}_{ij}t^{ij}_{ab}
\end{align}  

Last contribution form the energy expression to be resolved is:
\begin{align}
\bra{0}\hat{H}^{(1)}\ket{1_{\text{F12}}}=\bra{0}\bar{H}_{N}\ket{1_{\text{F12}}}+\bra{0}\hat{T}_{1}^{\dagger}\bar{H}_{N}\ket{1_{\text{F12}}}+\bra{0}\hat{T}_{2}^{\dagger}\bar{H}_{N}\ket{1_{\text{F12}}}
\end{align}   

Taking into consideration only terms that will be at most linear in amplitudes and GBC, this term will give same contributions as previous one.
By defining intermediate $\mathcal{V}$ as:
\begin{align}
\mathcal{V}^{ij}_{ij}=\bar{V}^{ij}_{ij}+\bar{V}^{ia}_{ij}t^{j}_{a}+\bar{V}^{aj}_{ij}t^{i}_{a}+(\bar{V}^{ab}_{ij}+\bar{C}^{ab}_{ij})t^{ij}_{ab}
\end{align}   

then the energy contribution in spin-free formalism is defined as:
\begin{align}
E_{(2)_{\overline{\text{F12}}}}=2\mathcal{V}^{ij}_{ij}+\mathcal{B}^{ij}_{ij}
\end{align}   



\end{document}
